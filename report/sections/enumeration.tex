\chapter{Introduction}
Le dénombrement est le principe de compter un nombre d'objets \\ \\

\section{Notation}
\subsection{Ensemble / Sous-ensemble}
Si on ne prend \emph{pas} en compte l'ordre de plusieurs éléments on les met entre \emph{accolades}, c'est un ensemble \\ \\
$\{a, b, c\} \subset E$

\subsection{Liste ordonnée}
Si on prend en compte l'ordre de plusieurs éléments on les met entre \emph{parenthèses} $(a, b, c)$ et on appelle ça une \emph{liste ordonnée d'éléments}. \\
Cette liste résulte du produit cartésiens d'ensembles $E_1 \times E_2 \times \dots \times E_n$
\paragraphb{Exemple}
$A = \{a, b, c\}$ et $B = \{0, 1, 2\}$ \\
$A \times B = \{(a, 0), (a, 1), (a, 2), (b, 0), (b, 1), (b, 2), (c, 0), (c, 1), (c, 2)\}.$

\pagebreak

\section{p-uplets}
\paragraphb{Exemple: Tirages successifs \emph{avec} remise}
$E$ est notre ensemble contenant tous nos résultats possibles

\begin{itemize}
  \item On pioche un premier jeton: $b$ et on le remet dans le sac
  \item On pioche un deuxième jeton: $a$ et on le remet dans le sac
  \item On pioche un troisième jeton: $a$ et on le remet dans le sac \\
\end{itemize}

On peut alors représenter ce résultat par $(b, a, a)$ \\ \\
\textbf{Les résultats possibles de cette expérience sont des listes de 3 éléments de $E$,
  \emph{avec} répétition d’éléments possible.}

\paragraphb{Attention} on respecte l'ordre $(a, b, c) \neq (c, b, a)$

\pagebreak

\section{Arrangements}
\paragraphb{Exemple: Tirages successifs \emph{sans} remise}
$E$ est notre ensemble contenant tous nos résultats possibles

\begin{itemize}
  \item On pioche un premier jeton: $b$ que l’on ne remet \textbf{pas} dans le sac
  \item On pioche un deuxième jeton: $a$ que l’on ne remet \textbf{pas} dans le sac
  \item On pioche un troisième jeton: $c$ que l’on ne remet \textbf{pas} dans le sac
\end{itemize}

On peut alors représenter ce résultat par $(b, a, c)$ \\ \\
\textbf{Les résultats possibles de cette expérience sont des listes de 3 éléments de $E$,
  \emph{sans} répétition d’éléments possible.}

\paragraph{En résumé}
\begin{itemize}
  \item Une liste de $3$ éléments sans répétition possible est appelée un arrangement de $3$ éléments
  \item Plus généralement, une liste de $p$ éléments sans répétition possible est appelée un \emph{arrangement de $p$ éléments de $E$}
\end{itemize}

\pagebreak

\section{Combinaisons}
\paragraphb{Exemple: Tirages \emph{simultanés}}
$E$ est notre ensemble contenant tous nos résultats possibles

\begin{itemize}
  \item On $3$ trois jetons en une seule fois: $a$, $d$ et $c$
\end{itemize}

On peut alors représenter le résultat entre accolades car on ne prend par en compte l'odre d'arrivé des jetons: \\ $\{ a, d, c \} = \{ c, d, a\}$ \emph{(l'ordre ne compte pas)} \\ \\
\textbf{Les résultats possibles de cette expérience sont un sous-ensemble de 3 éléments de $E$, on appelle ça une combinaison de $E$} \\
$\{a, d, c\} \subset E$

\pagebreak

\section{Permutations}
\paragraphb{Exemple: Tirage successifs sans remise}
C'est un cas particulier d'arrangement
Si l’on réalise autant de pioches sans remise qu’il y a de jetons dans le sac, on obtient alors une liste de tous les éléments de $E$ rangés dans un certain ordre. \\ \\
Cette liste est appelée une permutation de $E$ \\ \\
Par exemple : $(d, b, c, a)$ est une permutation des éléments de $E$ \\
Et: $(c, a, d, b)$ en est une autre \\ \\
Plus généralement un arrangement de $n$ éléments d’un ensemble $E$ à $n$ éléments est appelé permutation des éléments de $E$.
