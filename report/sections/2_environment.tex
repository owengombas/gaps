\chapter{Environnement}

\section{Swift}
Le choix s'est porté sur le langage d'Apple, Swift, pour la réalisation de ce projet. C'est un langage de programmation moderne, puissant et facile à apprendre. Il est conçu pour être sûr, performant et expressif. Swift est un langage de programmation multi-paradigme, qui supporte le paradigme impératif, le paradigme fonctionnel et le paradigme orienté objet.

Ce langage s'est donc imposé comme le choix idéal pour ce projet, car il permet de développer des applications performantes et facilement portables sur les différents systèmes d'exploitation d'Apple.

\section{SwiftUI}
SwiftUI est un framework de développement d'interface graphique pour les applications iOS, macOS, watchOS et tvOS. Il permet de créer des interfaces graphiques de manière déclarative, et de les mettre à jour automatiquement lorsqu'un changement est détecté. Il est possible de créer des interfaces graphiques en utilisant des composants prédéfinis, ou de créer ses propres composants.

\section{Playground}
Apple fournit un outil de développement nommé Playground, qui permet de développer et d'exécuter du code Swift. Il est possible de créer des pages de code, qui peuvent être exécutées séparément. De la manière qu'un notebook Jupyter permet de créer des cellules de code, un Playground permet de créer des pages de code. Ces pages peuvent être exécutées séparément, et les résultats sont affichés dans la console. Cela permet de tester des idées, de débugger du code, et de créer des exemples.

Cette fonctionnalité a été très utile afin d'effectuer les calculs nécessaires afin de trouver les meilleures heuristiques pour le projet.

